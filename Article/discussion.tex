\clearpage
\newpage



\section{Discussion}

Including geographic information on sample locations in the inference of ancestral relationships among  organisms is a major objective of population genetic studies (Mal\'ecot 1948, Cavalli-Sforza et al. 1994, Epperson 2003). Assuming that geographically close individuals are more likely to share ancestry than individuals at distant sites, we introduced two new  algorithms for estimating ancestry proportions using geographic information. Based on least-squares problems, the new algorithms combine matrix factorization approaches and spatial statistics to provide accurate estimates of individual ancestry coefficients and ancestral genotype frequencies. The two methods share many similarities, but they differ in the approximations they make in order to decrease algorithmic complexity.  More specifically, the AQP algorithm was based on quadratic programming, whereas the APLS algorithm was based on the spectral decomposition of the Laplacian matrix. The algorithmic complexity of APLS algorithm grows linearly with the number of individuals in the sample while the method has the same statistical accuracy as more complex algorithms. 


To measure the benefit of using spatial algorithms, we compared the statistical errors observed for spatial algorithms with those observed for non-spatial algorithms. The errors of spatial methods were lower than those observed  with non-spatial methods, and spatial algorithms allowed the detection of more subtle population structure. In addition, we implemented neutrality tests based on the spatial estimates of the $Q$ and $G$-matrices (Martins et al. 2016), and we observed that those tests had higher power to reject neutrality than those based on non-spatial approaches. Thus spatial information helped improving the detection of signatures of selective sweeps having occurred  in ancestral populations prior to admixture events. We applied the neutrality tests to perform a genome scan for selection in European ecotypes of the plant species {\it A. thaliana}. The genome scan provided evidence for selection at flowering-related genes {\it CIP4.1}, {\it FRI} and {\it DOG1} differentiating Fennoscandia from North-West Europe (Horton et al. 2012).

Estimation of ancestry coefficients using fast algorithms that extend non-spatial approaches -- such as {\tt structure} -- has been intensively discussed during the last years (Wollstein and Lao 2015). In these improvements, spatial approaches have received less attention than non-spatial approaches. In this study, we have proposed a conceptual framework for developing fast spatial ancestry estimation methods, and a suite of computer programs implements this framework in the {\tt R} program {\tt tester}. Our package provides an integrated pipeline for estimating and visualizing population genetic structure,  and for scanning genomes for signature of local adaptation. The algorithmic complexity of our algorithms allow their users to analyze samples including hundreds to thousands of individuals. For example, analyzing more than one thousand {\it A. thaliana} genotypes, each including more than 210k SNPs, took less than a few minutes using a single CPU. In addition, the algorithms have multithreaded versions that run on parallel computers by using multiple CPUs. The multithreaded algorithm, which is available from the {\tt R} program, allows using our programs in large-scale genomic sequencing projects. 

 
 
 



 























