
% use option [preprint] to remove info line at bottom
% journal options: aop,aap,aos,aoas,ssy
% natbib option: authoryear
\documentclass[12pt,aoas]{imsart} %% to review
%%\documentclass[aoas]{imsart}

%%%%%%%%%%%%%%%%%%%%%%%%%%%%%%%%%%%%%%%%%%%%%%%%%%%%%%%%%%%%%%%%%%%%%%%%%%%%%%%%
% Package
\usepackage{amssymb}
\usepackage{verbatim}
%\usepackage[francais,english]{babel}
\usepackage{graphicx}
\usepackage{lineno}
\RequirePackage{natbib}
\usepackage{hyperref}
%\usepackage[latin1]{inputenc}
%\usepackage[table]{xcolor}

% provide arXiv number if available:
%\arxiv{arXiv:0000.0000}

% put your definitions there:
\startlocaldefs
\endlocaldefs

\begin{document}

%% To review
\baselineskip 0.8cm
\linenumbers
%%

\begin{frontmatter}

% "Title of the paper"
\title{Fast inference of individual admixture coefficients using geographic data}
\runtitle{Fast inference of individual admixture coefficients}

\begin{aug}
\author{\fnms{Kevin} \snm{Caye}\thanksref{t1}\ead[label=e1]{kevin.caye@imag.fr}},
\author{\fnms{Flora} \snm{Jay}\thanksref{t2}\ead[label=e2]{flora.jay@lri.fr}},
\author{\fnms{Olivier} \snm{Michel}\thanksref{t1}\ead[label=e3]{olivier.michel@gipsa-lab.grenoble-inp.fr}},
\and
\author{\fnms{Olivier} \snm{Fran\c cois}\thanksref{t1}
\ead[label=e4]{olivier.francois@imag.fr}}

%% \thankstext{m1}{This work has been partially supported by the LabEx PERSYVAL-Lab (ANR-11-LABX-0025-01) funded by the French program Investissement d\rq{}Avenir}
%% \thankstext{m2}{Olivier Fran\c cois acknowledges support from Grenoble INP and from the Agence Nationale de la Recherche, project AFRICROP ANR-13-BSV7-0017}
\runauthor{K. Caye et al.}

\affiliation{Universit\'e Grenoble-Alpes\thanksmark{t1} and Universit\'e Paris Diderot\thanksmark{t2}}

\address{Kevin Caye and Olivier Fran\c cois\\
Universit\'e Grenoble-Alpes\\
Centre National de la Recherche Scientifique\\ 
TIMC-IMAG UMR 5525\\
Grenoble, 38042, France\\
\printead{e1}\\
\phantom{E-mail:\ }\printead*{e4}}

\address{Flora Jay\\
Université Paris Diderot\\
Centre National de la Recherche Scientifique\\
Eco-anthropologie et Ethnobiologie UMR 7206\\
Paris, 75013, France\\
\printead{e2}}

\address{Olivier Michel\\
Universit\'e Grenoble-Alpes\\
Centre National de la Recherche Scientifique\\ 
GIPSA-lab UMR 5216\\
Grenoble, 38042, France\\
\printead{e3}}


\end{aug}

\begin{abstract}
Accurately evaluating the distribution of genetic ancestry across geographic space is one of the main questions addressed by evolutionary biologists. This question has been commonly addressed through the application of Bayesian estimation programs allowing their users to estimate individual admixture proportions and allele frequencies among putative ancestral populations. Following the explosion of high-throughput sequencing technologies, several algorithms have been proposed to cope with computational burden generated by the massive data in those studies. In this context, incorporating geographic proximity in ancestry estimation algorithms is an open statistical and computational challenge. In this study, we introduce new algorithms that use geographic information to estimate ancestry proportions and ancestral genotype frequencies using population genetic data. Our algorithms combine matrix factorization methods and spatial statistics to provide estimates of ancestry matrices based on least-squares approximation.  We demonstrate the benefit of using spatial algorithms through extensive computer simulations, and we provide an example of application of our new algorithms to a set of spatially referenced samples for the plant species {\it Arabidopsis thaliana}.  Without lost of statistical accuracy, the new algorithms exhibit runtimes that are much shorter than those observed for previously developed spatial methods. Our algorithms are implemented in the {\tt R} package, {\tt tess3r}, which is available from \url{https://github.com/BioShock38/TESS3_encho_sen}. 
\end{abstract}

\begin{keyword}
\kwd{Ancestry Estimation Algorithms}
\kwd{Genotypic Data}
\kwd{Geographic Data}
\kwd{Fast Algorithms}
\end{keyword}

\end{frontmatter}

\section{Introduction}

 High-throughput sequencing technologies have enabled studies of genetic ancestry for model and non-model species at an unprecedented pace. In this  context, ancestry estimation algorithms are important for demographic analysis, medical genetics, conservation and landscape genetics~\citep{Pritchard2000, Tang2005, Schraiber2015, Segelbacher2010, Francois2015}.  With increasingly large data sets, Bayesian approaches to the inference of population structure, exemplified by the computer program {\tt structure} \citep{Pritchard2000}, have been replaced by approximate algorithms that run several orders faster than the original version~\citep{Tang2005, Alexander2011, Frichot2014, Raj2014}.  Considering $K$ ancestral populations or genetic clusters, those algorithms estimate ancestry coefficients following two main directions: model-based and model-free approaches. In model-based approaches, a likelihood function is defined for the matrix of ancestry coefficients, and estimation is performed by maximizing the logarithm of the likelihood function. For {\tt structure} and derived models, model assumptions include linkage equilibrium and Hardy-Weinberg equilibrium in ancestral populations. The first approximation to the original algorithm was based on an expectation-minimization algorithm~\citep{Tang2005}, and more recent likelihood algorithms are implemented in the programs {\tt admixture}  and {\tt faststructure} \citep{Alexander2011, Raj2014}. In model-free approaches, ancestry coefficients are estimated by using least-squares methods or factor analysis. Model-free methods make no assumptions about the biological processes that have generated the data. To estimate ancestry matrices, \cite{Engelhardt2010} proposed to use sparse factor analysis, \cite{Frichot2014} used sparse non-negative matrix factorization algorithms, and \cite{Popescu2014} used kernel-principal component analysis. Least-squares methods accurately reproduce the results of likelihood approaches under the model assumptions of those methods~\citep{Frichot2014, Popescu2014}.  In addition, model-free methods provide approaches that are valid when the assumptions of likelihood approaches are not met. Model-free methods are generally faster than model-based methods. 
   
  Among model-based approaches to ancestry estimation, an important class of methods have improved the Bayesian model of {\tt structure} by incorporating geographic data through spatially informative prior distributions~\citep{Chen2007, Corander2008}. Under isolation-by-distance patterns~\citep{Wright1943, Malecot1948}, spatial algorithms provide more robust estimates of population structure than non-spatial algorithms which can lead to biased estimates of the number of clusters~\citep{Durand2009}.  Some Bayesian methods are based on Markov chain Monte Carlo algorithms which are computer-intensive~\citep{Francois2010}. Recent efforts to improve the inference of ancestral relationships in a geographical context have mainly focused on the localization of recent ancestors~\citep{Baran2013, Lao2014, Yang2014}. In these applications, spatial information is used in a predictive framework that assigns ancestors to putative geographic origins. While fast geographic estimation of individual ancestry proportions has been proposed previously~\citep{Caye2016}, there is a growing need to develop individual ancestry estimation algorithms that reduce computational cost in a geographically explicit framework. 

In this study, we present two new algorithms for the estimation of ancestry matrices based on geographic and genetic data. The new algorithms solve a least squares optimization problem as defined by~\cite{Caye2016}, based on Alternating Quadratic Programming  (AQP) and Alternating Projected Least Squares (APLS). While AQP algorithms have a well-established theoretical background~\citep{Bertsekas1995}, this is not the case of APLS algorithms. Using coalescent simulations, we provide evidence that the estimates computed by APLS algorithms are good approximations to the solutions of AQP algorithms. In addition, we show that the performances of APLS algorithms scale with the dimensions of modern data sets. We discuss the application of our algorithms to data from European ecotypes of {\it Arabidopsis thaliana}, for which individual genomic an geographic data are available~\citep{Horton2012}. 




\clearpage
\newpage

\section{New methods}

%% Présenter le problème et la factorisation matricielle
%%
%%

In this section we present two new algorithms for estimating individual admixture coefficients and ancestral genotype frequencies assuming $K$ ancestral populations. In addition to genotypes, the new algorithms require individual geographic coordinates of sampled individuals.

\paragraph{$Q$ and $G$-matrices} Consider a genotypic matrix, {\bf Y}, recording data for $n$ individuals at $L$ polymorphic loci for a $p$-ploid species (common values for $p$ are $p = 1,2$). For autosomal SNPs in a diploid organism, the genotype at locus $\ell$  is an integer number, 0, 1 or 2, corresponding to the number of reference alleles at this locus. In our algorithms, disjunctive forms are used  to encode each genotypic value as the indicator of a heterozygote or a homozygote locus (Frichot et al. 2014). For a diploid organism each genotypic value $,0,1,2$ is encoded as $100$, $010$ and $001$. For $p$-ploid organisms, there are $(p+1)$ possible genotypic values at each locus, and each value corresponds to a unique disjunctive form. While our focus is on SNPs, the algorithms presented in this section extend to multi-allelic loci without loss of generality. 
Moreover, the method can be easily extended to genotype likelihoods by using the likelihood to encode each genotypic value~\citep{Korneliussen2014}.

Our algorithms provide statistical estimates for the matrix ${\bf Q} \in \mathbb{R}^{K \times n}$ which contains the admixture coefficients, ${\bf Q}_{i,k}$, for each sampled individual, $i$, and each ancestral population, $k$. The algorithms also provide estimates for the matrix ${\bf G} \in \mathbb{R}^{(p+1)L \times K}$, for which the entries, ${\bf G}_{(p+1)\ell + j, k}$, correspond to the frequency of genotype $j$ at locus $\ell$ in population $k$. Obviously, the $Q$ and $G$-matrices must satisfy the following set of probabilistic constraints 

$$
\quad {\bf Q},{\bf G} \geq 0 \, , \quad  \sum_{k=1}^K {\bf Q}_{i,k} = 1 \, , \quad \sum_{j=0}^p {\bf G}_{(p+1)\ell + j, k} = 1 \, , \quad j = 0,1,\dots, p,
$$
for all $i, k$ and $\ell$. Using disjunctive forms and the law of total probability, estimates of {\bf Q} and {\bf G} can be obtained by factorizing the genotypic matrix as follows ${\bf Y}$=${\bf Q}\,{\bf G}^T$~\citep{Frichot2014}. Thus the inference problem can be solved by using constrained nonnegative matrix factorization methods~\citep{Lee1999, Cichocki2009}. In the sequel, we shall use the notations  $\Delta_Q$ and $\Delta_G$ to represent the sets of probabilistic constraints put on the {\bf Q} and {\bf G} matrices respectively. 


 \paragraph{Geographic weighting} Geography is introduced in the matrix factorization problem by using weights for each pair of sampled individuals. The weights impose regularity constraints on ancestry estimates over geographic space. The definition of geographic weights is based on the spatial coordinates of the sampling sites, $(x_i)$. Samples close to each other are given more weight than samples that are far apart. The computation of the weights starts with building a complete graph from the sampling sites. Then the weight matrix is defined as follows

$$
w_{ij} = \exp( - {\rm dist}( x_i, x_j )^2/ \sigma^2),
$$
\noindent where dist$( x_i, x_j )$ denotes the geodesic distance between sites $x_i$ and  $x_j$, and $\sigma$ is a range parameter. Values for the range parameter can be investigated by using spatial variograms~\citep{Cressie1993}. 
To evaluate variograms, we extend the univariate variogram to genotypic data as follows

\begin{equation}
\gamma(h) = \frac{1}{2 |N(h)|} \sum_{i,j \in N(h)} \frac{1}{L} \sum_{l} |Y_{i,\ell} - Y_{j,\ell}|,
\label{eq:gamma}
\end{equation}
\noindent where $N(h)$ is defined as the set of individuals separated by geographic distance $h$. 
% The function $\gamma$ can be approximated as follows 
% \begin{equation}
% \hat{\gamma}(h) = {C}_0 \frac{1}{2 |N(h)|} \sum_{i,j \in N(h)} \| {\bf Q}_{i,.} - {\bf Q}_{j,.}\|^2 + {C}_1,~~~C_0, C_1 > 0,
% \label{eq:gammahat}
% \end{equation}
% \noindent where $\|  u \|^2$ is the squared norm of the vector $u$, and ${\bf Q}_{i,.}$ is the $i$th row of the admixture matrix ${\bf Q}$. Arguments that justify this approximation are given in Appendix~\ref{app:approx}. 
In applications, computing and visualizing the $\gamma$ function  provides useful information on the level of spatial autocorrelation between individuals in the data. 


Next, we introduce the {\it Laplacian matrix} associated with the geographic weight matrix, {\bf W}. The Laplacian matrix is defined as ${\bf \Lambda}$ = {\bf D} $-$ {\bf W}  where  {\bf D} is a diagonal matrix with entries
${\bf D}_{i,i} = \sum_{j = 1}^n  {\bf W}_{i,j}$,  for  $i = 1, \dots, n$~\citep{Belkin2003}. Elementary matrix algebra shows that~\citep{DengCai2011}

$$
 {\rm Tr} ({\bf Q}^T {\bf \Lambda} {\bf Q})  = \frac12 \sum_{i,j = 1}^n  w_{ij}  \|   {\bf Q}_{i,.}  - {\bf Q}_{j,.} \|^2 \, .
$$
In our approach, assuming that geographically close individuals are more likely to share ancestry than individuals at distant sites is thus equivalent to minimizing the quadratic form ${\cal C}({\bf Q}) ={\rm Tr} ({\bf Q}^T {\bf \Lambda} {\bf Q})$ while estimating the matrix ${\bf Q}$. 

\paragraph{Least-squares optimization problems} Estimating the matrices ${\bf Q}$ and ${\bf G }$ from the observed genotypic matrix ${\bf Y}$ is performed through solving an optimization problem defined as follows~\citep{Caye2016}

\begin{equation}
\begin{aligned}
& \underset{Q, G}{\text{min}}
& & {\rm LS}({\bf Q}, {\bf G}) =   \|  {\bf Y} - {\bf QG}^T \|^2_{\rm F} +  \alpha ' \frac{(p+1)L}{K \lambda_{\max}} {\cal C}({\bf Q}) , \\
& \text{s.t.} & &  {\bf Q} \in \Delta_Q , \\
& & &  {\bf G} \in \Delta_G . \\
\end{aligned}
\label{eq:LS}
\end{equation}
 \noindent The notation $\|  {\bf M}  \|_{\rm F}$ denotes the Frobenius norm of a matrix, {\bf M}. The regularization term is normalized by $(p+1)L/K \lambda_{\max}$, where $\lambda_{\max}$ is the largest eigenvalue of the Laplacian matrix. With this normalization, both terms of the optimization problem~\eqref{eq:LS} are given the same order of magnitude. The regularization parameter $\alpha ' $ controls the regularity of ancestry estimates over geographic space.  Large values of $\alpha ' $ imply that ancestry coefficients have similar values for nearby individuals, whereas small values ignore spatial autocorrelation in observed allele frequencies. In the rest of the article, we will use $\alpha ' = 1$ and $\alpha = (p+1)L/K \lambda_{\max}$. Using the least-squares approach, the number of ancestral populations, $K$, can be chosen after the evaluation of a cross-validation criterion for each $K$~\citep{Alexander2011, Frichot2014, Frichot2015}.


\paragraph{The Alternating Quadratic Programming (AQP) method} Because the poly\-edrons $\Delta_Q$ and  $\Delta_G$ are convex sets and the LS function is convex with
respect to each variable ${\bf Q}$ or ${\bf G}$ when the other one is fixed, the problem~\eqref{eq:LS} is amenable to the application of block coordinate descent~\citep{Bertsekas1995}. The APQ algorithm starts from initial values for the $G$ and $Q$-matrices, and alternates two steps. The first step computes the matrix {\bf G} while  {\bf Q} is kept fixed, and the second step permutates the roles of {\bf G} and {\bf Q}.  Let us assume that {\bf Q} is fixed and write {\bf G} in a vectorial form, $g = {\rm vec({\bf G})} \in \mathbb{R}^{K(p + 1)L}$. The first step of the algorithm actually solves the following quadratic programming subproblem. Find  

\begin{equation}
\begin{aligned}
g^\star = \underset{g \in \Delta_G}{\arg \min}  ( -2  v^T_Q \, g + g^T {\bf D}_Q g ) \, ,  
\end{aligned}
\label{eq:AQPg}
\end{equation}
\noindent where ${\bf D}_Q = {\bf I}_{(p+1)L} \otimes {\bf Q}^T {\bf Q}$ and $v_Q = {\rm vec}({\bf Q}^T {\bf Y})$. Here, $\otimes$ denotes the Kronecker product and ${\bf I}_d$ is the identity matrix with $d$ dimensions. Note that the block structure of the matrix ${\bf D}_Q$ allows us to decompose the subproblem~\eqref{eq:AQPg} into $L$ independent quadratic programming problems with $K(p + 1)$ variables. Now, consider that {\bf G} is the value obtained after the first step of the algorithm, and write {\bf Q} in a vectorial form, $q = {\rm vec({\bf Q})} \in \mathbb{R}^{nK}$. The second step solves the following quadratic programming subproblem. Find

\begin{equation}
\begin{aligned}
q^\star = \underset{q \in \Delta_Q}{\arg \min} ( -2 v^T_G \, q + q^T {\bf D}_G q ) \,  ,
\end{aligned}
\label{eq:AQPq}
\end{equation}

\noindent where ${\bf D}_G = {\bf I}_{n} \otimes {\bf G}^T {\bf G } + \alpha {\bf \Lambda}  \otimes {\bf I}_K$ and $v_G = {\rm vec}({\bf G}^T{\bf  Y}^T)$. Unlike subproblem~\eqref{eq:AQPg}, subproblem~\eqref{eq:AQPq} can not be decomposed into smaller problems. Thus, the computation of the second step of the AQP algorithm implies to solve a quadratic programming problem with $nK$ variables which can be problematic for large samples ($n$ is the sample size). 
The AQP algorithm is described in details in Appendix~\ref{algo:aqp}. For AQP, we have the following convergence result.
\begin{thm}
\label{th}
	The AQP algorithm converges to a critical point of problem~\eqref{eq:LS}.
\end{thm}
\begin{proof}
The quadratic convex functions defined in subproblems~\eqref{eq:AQPg} and~\eqref{eq:AQPq} have finite lower bounds. The convex sets $\Delta_Q$ and $\Delta_G$ are not empty sets, and they are compact sets. Thus the sequence generated by the AQP algorithm is well-defined, and has limit points.
According to Corollary 2 of ~\cite{Grippo2000}, we conclude that
the AQP algorithm converges to a critical point of problem~\eqref{eq:LS}.
\end{proof}

\paragraph{Alternating Projected Least-Squares (APLS)} In this paragraph, we introduce an APLS estimation algorithm which approximates the solution of problem~\eqref{eq:LS}, and reduces the complexity of the AQP algorithm. The APLS algorithm starts from initial values of the $G$ and $Q$-matrices, and alternates two steps. The matrix {\bf G} is computed  while  {\bf Q} is kept fixed, and {\it vice versa}. Assume that the matrix {\bf Q} is known. The first step of the APLS algorithm solves the following optimization problem. Find 
\begin{equation}
{\bf G}^\star = \arg \min  \|  {\bf Y} - {\bf QG}^T \|^2_{\rm F} \, .
\end{equation}
This operation can be done by considering $(p+1)L$ (the number of columns of ${\bf Y}$) independent optimization problems running in parallel. The operation is followed by a projection of ${\bf G}^\star$ on the polyedron of constraints, $\Delta_G$. For the second step, assume that {\bf G} is set to the value obtained after the first step is completed. We compute the eigenvectors, {\bf U}, of the Laplacian matrix, and we define the diagonal matrix ${\bf \Delta}$ formed by the eigenvalues of ${\bf \Lambda}$ (The eigenvalues of ${\bf \Lambda}$ are non-negative real numbers). According to the spectral theorem, we have

$$
{\bf \Lambda} = {\bf U}^T {\bf \Delta} {\bf U} \, .
$$
\noindent  After this operation, we project the data matrix {\bf Y} on the basis of eigenvectors as follows

$$
{\rm proj} ({\bf Y}) = {\bf U}{\bf Y} \, , 
$$
\noindent and, for each individual, we solve the following optimization problem

\begin{equation}
q_i^\star = \arg \min  \|  {\rm proj} ({\bf Y})_i  - {\bf G}^Tq \|^2 + \alpha \lambda_i \| q \|^2  \, ,
\label{eq:APSLq}
\end{equation}
\noindent where  proj({\bf Y}$)_i$ is the $i$th row of the projected data matrix, proj({\bf Y}), and $\lambda_i$ is the $i$th eigenvalue of ${\bf \Lambda}$. The solutions, $q_i$, are then concatenated into a matrix, ${\rm conc}(q)$, and ${\bf Q}$ is defined as the projection of the matrix ${\bf U}^T {\rm conc}(q)$ on the polyedron $\Delta_Q$. The complexity of step~\eqref{eq:APSLq} grows linearly with $n$, the number of individuals. While the theoretical convergence properties of AQP algorithms are lost for APLS algorithms, the APLS algorithms are expected to be good approximations of AQP algorithms. The APLS algorithm is described in details in Appendix~\ref{algo:apls}.

\paragraph{Comparison with {\tt tess3}}  The algorithm implemented in a previous version of {\tt tess3} also provides approximation of of solution of~\eqref{eq:LS}. The {\tt tess3} algorithm first computes a Cholesky decomposition of the Laplacian matrix. Then, by a change of variables, the least-squares problem is transformed into a sparse nonnegative matrix factorization problem~\citep{Caye2016}.  Solving the sparse non-negative matrix factorization problem relies on the application of existing methods~\citep{Kim2011, Frichot2014}. The methods implemented in {\tt tess3} have an algorithmic complexity that increases linearly with the number of loci and the number of clusters. They lead to estimates that accurately reproduce those of the Monte Carlo algorithms implemented in the Bayesian method {\tt tess} 2.3~\citep{Caye2016}. Like for the AQP method, the {\tt tess3} previous algorithms have an algorithmic complexity that increases quadratically with the sample size. 




\paragraph{Ancestral population differentiation statistics and local adaptation scans} Assuming $K$ ancestral populations, the $Q$ and $G$-matrices  obtained from the AQP and from the APLS algorithms were used to compute single-locus estimates of a population differentiation statistic similar to $F_{\rm ST}$~\citep{Martins2016}, as follows

$$
F^{Q}_{\rm ST} = 1 - \sum_{k=1}^K  q_k \frac{f_k (1-f_k)}{f(1-f)} \, ,
$$

\noindent where $q_k$ is the average of ancestry coefficients over sampled individuals, $q_k = \sum_{i =1}^n q_{ik}/n$, for the cluster $k$, $f_k$ is the ancestral allele frequency in population $k$ at the locus of interest, and $f = \sum_{k = 1}^K q_k f_k$ (Martins et al. 2016). The locus-specific statistics were used to perform statistical tests of neutrality at each locus, by comparing the observed values to their expectations from the genome-wide background. The test was based on the squared $z$-score statistic, $z^2 = (n-K) F^{Q}_{\rm ST}/(1 - F^{Q}_{\rm ST})$, for which a  chi-squared distribution with $K-1$ degrees of freedom was assumed under the null-hypothesis~\citep{Martins2016}. The calibration of the null-hypothesis was achieved by using genomic control to adjust the test statistic for background levels of population structure~\citep{Devlin1999, Francois2016}. After recalibration of the null-hypothesis, the control of the false discovery rate was achieved by using the Benjamini-Hochberg algorithm~\citep{Benjamini1995}.


\paragraph{{\tt R} package} We implemented the AQP and APLS algorithms in the {\tt R} package {\tt tess3r}, available from Github and submitted to the Comprehensive R Archive Network (R Core Team, 2016).  


\section{Simulated and real data sets}
\paragraph{Coalescent simulations} We used the computer program {\tt ms} to perform coalescent simulations of neutral and outlier SNPs under spatial models of admixture~\citep{Hudson2002}. Two ancestral populations were created from the simulation of Wright\rq{}s two-island models. The simulated data sets contained admixed genotypes for $n$ individuals for which the admixture proportions varied continuously along a longitudinal gradient~\citep{Durand2009, Francois2010}. In those scenarios, individuals at each extreme of the geographic range were representative of their population of origin, while individuals at the center of the range shared intermediate levels of ancestry in the two ancestral populations~\citep{Caye2016}. For those simulations, the $Q$ matrix, ${\bf Q}_0$, was entirely described by the location of the sampled individuals.


Neutrally evolving ancestral chromosomal segments were generated by simulating DNA sequences with an effective  population size $N_0 = 10^6$ for each ancestral population. The mutation rate per bp and generation was set to $\mu = 0.25 \times 10^{-7}$, the recombination rate per generation was set to $r = 0.25 \times 10^{-8}$, and the parameter $m$ was set to obtained neutral levels of $F_{\rm ST}$ ranging between values of $0.005$ and $0.10$. The number of base pairs for each DNA sequence was varied between 10k to 300k to obtain numbers of polymorphic locus ranging between 1k and 200k after filtering out SNPs with minor allele frequency lower than 5$\%$.  To create SNPs with values in the tail of the empirical distribution of $F_{\rm ST}$,  additional ancestral chromosomal segments were generated by simulating DNA sequences with a migration rate $m_s$ lower than $m$. The simulations reproduced the reduced levels of diversity and the increased levels of differentiation expected under hard selective sweeps occurring at one particular chromosomal segment in ancestral populations~\citep{Martins2016}.  For each simulation, the sample size  was varied in the range $n =$ 50-700.

 We compared the AQP and APLS algorithm estimates with those obtained with the {\tt tess3} algorithm.  Each program was run 5 times.  Using $K = 2$ ancestral populations, we computed the root mean squared error (RMSE) between the estimated and known values of the $Q$-matrix, and between  the estimated and known values of the $G$-matrix. 
To evaluate the benefit of spatial algorithms, we compared the statistical errors of APLS algorithms to the errors obtained with {\tt snmf} method that reproduces the outputs of the {\tt structure} program accurately~\citep{Frichot2014,Frichot2015}.  To quantify the performances of neutrality tests as a function of ancestral and observed levels of $F_{\rm ST}$, we used the area under the precision-recall curve (AUC) for several values of the selection rate.  Subsamples from a real data set were used to perform a runtime analysis of the AQP and APLS algorithms ({\it A. thaliana} data, see below). Runtimes were evaluated by using a single computer processor unit Intel Xeon 2.0 GHz.

\paragraph{Application to European ecotypes of {\it Arabidopsis  thaliana}} We used  the APLS algorithm to survey spatial population genetic structure and to investigate the molecular basis of adaptation  by considering SNP data from 1,095  European ecotypes of the plant species {\it A. thaliana} (214k SNPs, \cite{Horton2012}). The cross-validation criterion was used to evaluate the number of clusters in the sample, and a statistical analysis was performed to evaluate the range of the variogram from the data. We used {\tt R} functions of the {\tt tess3r} package to display interpolated admixture coefficients on a geographic map of Europe (R Core team 2016). A gene ontology enrichment analysis using the software AMIGO~\citep{Carbon2009} was performed in order to evaluate which molecular functions and biological processes might be involved in local adaptation in Europe.








\clearpage
\newpage


\section{Results}



\paragraph{Statistical errors.}  We used  coalescent simulations of neutral polymorphisms under spatial models of admixture to compare the statistical errors of the AQP and APLS
estimates with those of the {\tt tess3} algorithm. The ground truth for the $Q$-matrix (${\bf Q}_0$)  was computed from the mathematical model for admixture proportions used to generate the data. For the $G$-matrix, the  ground truth matrix (${\bf G}_0$) was computed from the empirical genotype frequencies in the 2 population samples before the admixture event.  The root mean squared errors (RMSE) for the ${\bf Q}$ and ${\bf G}$ estimates decreased as the sample size and as the number of loci increased (Figure 1). For all algorithms, the statistical errors were generally small when the number of loci was greater than $10k$ SNPs. Those results provided evidence that the three algorithms produced equivalent estimates of the matrices ${\bf Q}_0$ and ${\bf G}_0$. The results also provided a formal check that the APLS and {\tt tess3} algorithms converged to the same estimates as those obtained after the application of  the AQP algorithm, which is guaranteed to converge mathematically.   


\paragraph{The benefit of including spatial information in algorithms.}    Using  neutral coalescent simulations of spatial admixture, we compared the statistical estimates obtained from a spatial algorithm (APLS) and a non-spatial algorithm (sNMF, Frichot et al. 2014).  For various levels of ancestral population differentiation, estimates obtained from the spatial algorithm more accurate than for those obtained using non-spatial approaches (Figure 2). For the larger samples, much finer population structure was detected with the spatial method than with the non-spatial algorithm (Figure 2D). 

In simulations of outlier loci, we used the area under the precision-recall curve (AUC) for quantifying the performances of tests based on the estimates of the ancestry matrices, {\bf Q} and {\bf G}. In addition, we computed AUCs for $F_{\rm ST}$-based neutrality tests using the simulated ancestral genotypes. As they represented the maximum reachable values, AUCs based on truly ancestral genotypes were always higher than those obtained for tests based on reconstructed matrices. For all values of the relative  selection intensity, AUCs were higher for spatial methods than for non-spatial methods, (Figure 3, the relative selection intensity is the ratio of migration rates at neutral and adaptive loci). For high selection intensities, the performances of tests based on estimates of ancestry matrices were close to the optimal values reached  by tests based on true ancestral frequencies. These results provided evidence that including spatial information in ancestry estimation algorithms improves the detection of signatures of hard selective sweeps having occurred in unknown ancestral populations. 

\paragraph{Runtime and convergence analyses.} We subsampled a large SNP data set for {\it A. thaliana} ecotypes  to compare the convergence properties and runtimes of the {\tt tess3}, AQP, and APLS algorithms. In those experiments, we used $K = 6$ ancestral populations, and replicated 5 runs for each simulation. For $n = 100-600$ individuals ($L = 50k$ SNPs), the APLS algorithm required more iterations (25 iterations)  than the AQP algorithm (20 iterations) to converge to its solution  (Figure 4). This was less than for {\tt tess3} (30 iterations).  For $L = 10-200k$ SNPs ($n = 150$ individuals), similar results were observed. For $50k$ SNPs, the runtimes were significantly lower for the APLS algorithm than for the {\tt tess3} and AQP algorithms. For $L = 50k$ SNPs and $n = 600$ individuals, it took on average 0.956 min (100 min) for the APLS (AQP) algorithm to compute ancestry estimates. For {\tt tess3}, the runtime was on average 66.3 min. For $L = 100k$ SNPs and $n = 150$ individuals, it took on average 0.628 min (8.97 min) for the APLS (AQP) algorithm to compute ancestry estimates. For {\tt tess3}, the runtime was on average 1.27 min.  For those values of $n$ and $L$, the APLS algorithm implementation ran about 2 to 100 times faster than the other algorithm implementations.
 
 
\paragraph{Application to European ecotypes of {\it Arabidopsis  thaliana}.} We used  the APLS algorithm to survey spatial population genetic structure and perform a genome scan for adaptive alleles in European ecotypes of the plant species {\it A.  thaliana}. The cross-entropy criterion decreased rapidly from $K=1$ to $K=3$ clusters,  indicating  that  there were three 
main ancestral groups in Europe, corresponding to geographic regions in Western Europe, Eastern and Central Europe and Northern Scandinavia. For $K$ greater than four, the values of the cross-entropy criterion decreased in a slower way, indicating that subtle substructure resulting from complex historical isolation-by-distance  processes could also be detected (Figure 5). 
The kriging analysis provided an approximate range of  $r = 1.5e^2$ km for the spatial variogram (Figure 5).
%, which was consistent with the default parameter set for the geographic weights in the Laplacian matrix. 
Figure 6 displays a $Q$-matrix estimate  interpolated on a geographic map of Europe for $K = 6$ ancestral groups. The estimated admixture coefficients provided clear evidence for the clustering of the ecotypes in spatially homogeneous genetic groups. 

\paragraph{Targets of selection in {\it A.  thaliana} genomes.}  Tests based on the $F^Q_{\rm ST}$  statistic were applied to the 1,095 ecotype data to reveal new targets of natural selection in the {\it A. thaliana} genome. {\it A. thaliana} occurs in a broad variety of habitats, and local adaptation to the environment is acknowledged to be important in shaping its genetic diversity through space (Hancock et al. 2011, Fournier-Level et al. 2011). 
APLS algorithm was run on 1,095 European lines of {\it A. thaliana} with $K=6$ ancestral populations and $\sigma = 1.5$ for the range parameter. After controlling the FDR at the level $1\%$, the program produced a list of 12,701 candidate SNPs, including linked loci and representing 0.3\% of the total number of loci. 
%We observed a fold enrichment of 1.000152 in the targeted SNPs were found in exomic sequences (A VEP analysis indicated  that blabla). 
 
 The top 100 candidates included SNPs in the flowering-related genes SHORT VEGETATIVE PHASE (SVP), COP1-interacting protein 4.1 (CIP4.1) and FRIGIDA (FRI) with $p{\rm -values}$ of 0. 
 These genes were detected by previous scans for selection on this dataset (Horton et al. 2012).
 %These genes were previously reported as being involved in biological processes related to heat stress and defence response 
 
 We performed a gene ontology enrichment analysis using AmiGO in order to evaluate which biological functions might be involved in local adaptation in Europe (Carbon et al. 2009). 
 We found a significant over representation of genes involved in cellular processes (fold enrichment of 1.06 and $p{\rm -value}$ with Bonferonni correction of $2.15e^{-2}$).
% We found significant enrichment in molecular functions linked to catalytic activity (catalysis of biochemical reaction at physiological temperatures, GO:0003824, P = 1.6e-8) and hydrolase activity (G0:0016787, P = 2.7e-6).


% 
% HORTON:
% 
% "flowering-related genes" 
% 
% SHORT VEGETATIVE PHASE (SVP), a MADS box gene that negatively regulates the transition to flowering
% 
% (Differentiating Fennoscandia and Eastern Europe/Russia)
% 
% COP1-interacting protein 4.1 (CIP4.1)
% 
% FRIGIDA (FRI)
% 
% FLOWERING LOCUS C (FLC),
% 
% DELAY OF GERMINATION 1 (DOG1)
% 
% (Differentiating Fennoscandia from North-West Europe)




\clearpage
\newpage



\section{Discussion}
Including geographic information on sample locations in the inference of ancestral relationships among  organisms is a major objective of population genetic studies~\citep{Malecot1948, Cavalli-Sforza1994, Epperson2003}. Assuming that geographically close individuals are more likely to share ancestry than individuals at distant sites, we introduced two new  algorithms for estimating ancestry proportions using geographic information. Based on least-squares problems, the new algorithms combine matrix factorization approaches and spatial statistics to provide accurate estimates of individual ancestry coefficients and ancestral genotype frequencies. The two methods share many similarities, but they differ in the approximations they make in order to decrease algorithmic complexity.  More specifically, the AQP algorithm was based on quadratic programming, whereas the APLS algorithm was based on the spectral decomposition of the Laplacian matrix. The algorithmic complexity of APLS algorithm grows linearly with the number of individuals in the sample while the method has the same statistical accuracy as more complex algorithms. 


To measure the benefit of using spatial algorithms, we compared the statistical errors observed for spatial algorithms with those observed for non-spatial algorithms. The errors of spatial methods were lower than those observed  with non-spatial methods, and spatial algorithms allowed the detection of more subtle population structure. In addition, we implemented neutrality tests based on the spatial estimates of the $Q$ and $G$-matrices~\citep{Martins2016}, and we observed that those tests had higher power to reject neutrality than those based on non-spatial approaches. Thus spatial information helped improving the detection of signatures of selective sweeps having occurred  in ancestral populations prior to admixture events. We applied the neutrality tests to perform a genome scan for selection in European ecotypes of the plant species {\it A. thaliana}. The genome scan confirmed the evidence for selection at flowering-related genes {\it CIP4.1}, {\it FRI} and {\it DOG1} differentiating Fennoscandia from North-West Europe~\citep{Horton2012}.

Estimation of ancestry coefficients using fast algorithms that extend non-spatial approaches -- such as {\tt structure} -- has been intensively discussed during the last years~\citep{Wollstein2015}. In these improvements, spatial approaches have received less attention than non-spatial approaches. In this study, we have proposed a conceptual framework for developing fast spatial ancestry estimation methods, and a suite of computer programs implements this framework in the {\tt R} program {\tt tess3r}. Our package provides an integrated pipeline for estimating and visualizing population genetic structure,  and for scanning genomes for signature of local adaptation. The algorithmic complexity of our algorithms allow their users to analyze samples including hundreds to thousands of individuals. For example, analyzing more than one thousand {\it A. thaliana} genotypes, each including more than 210k SNPs, took less than a few minutes using a single CPU. In addition, the algorithms have multithreaded versions that run on parallel computers by using multiple CPUs. The multithreaded algorithm, which is available from the {\tt R} program, allows using our programs in large-scale genomic sequencing projects. 



\clearpage
\newpage

%\section*{Figure legends}

\begin{center}
\includegraphics[width=\textwidth]{../Figure1/Figures/figure1.pdf}
\end{center}
\noindent{\bf Figure 1.} {\bf Root Mean Squared Errors (RMSEs) for the $Q$ and $G$ matrix estimates.} Simulations of spatially admixed populations. A-B) Statistical errors for APLS, AQP and {\tt tess3} estimates as a function of the sample size, $n$ ($L \sim 10^4$). C-D) Statistical errors for APLS, AQP and {\tt tess3} estimates as a function of the number of loci, $L$ ($n = 200$).

\clearpage
\newpage


\begin{center}
\includegraphics[width=\textwidth]{../Figure2/Figures/figure2.pdf}
\end{center}
\noindent{\bf Figure 2.} {\bf Root Mean Squared Errors (RMSEs) for the $Q$ estimates.} Simulations of spatially admixed populations for several values of fixation index ($F_{\rm ST}$) between ancestral populations. Ancestral populations are simulated with Wright’s two-island models and the fixation index is defined as $1 / (1 + 4 N_0 m)$ where $m$ is the migration rate and $N_0$ the effective population size. The statistical errors for sNMF and APLS is represented as a function of the $F_{\rm ST}$ between ancestral populations.

\clearpage
\newpage

\begin{center}
\includegraphics[width=\textwidth]{../Figure3/Figures/figure3.pdf}
\end{center}
\noindent{\bf Figure 3.} {\bf Area under the precision-recall curve (AUC)}. Neutrality tests applied to simulations of spatially admixed populations. AUCs for tests based on $F_{\rm ST}$ with the true ancestral populations,  spatial ancestry estimates computed with APLS algorithms, non-spatial ({\tt structure}-like) ancestry estimates computed with the {\tt snmf} algorithm. The relative intensity of selection in ancestral populations, defined as the ratio $m/m_s$, was varied in the range $1-160$.


\clearpage
\newpage

\begin{center}
\includegraphics[width=\textwidth]{../Figure4/Figures/figure4.pdf}
\end{center}
\noindent{\bf Figure 4.} {\bf Number of iterations and runtimes for the AQP, APLS and {\tt tess3} algorithm implementations}. A-B)   Total number of iterations before an algorithm reached a steady solution. C-D) Runtime for a single iteration (seconds). The number of SNPs was kept fixed to $L = 50$k in A and C. The number of individuals was kept fixed to $n = 150$ in B and D.


\clearpage
\newpage

\begin{center}
\includegraphics[width=\textwidth]{../Figure5/Figures/figure5.pdf}
\end{center}
\noindent{\bf Figure 5.} {\bf Parameter assessment for the APLS algorithm}. A) Empirical variogram of the {\it A. thaliana} genetics matrix. The red vertical line shows the range $1.5$. B) Cross validation error as function of the number of ancestry coefficients $K$.

\clearpage
\newpage

\begin{center}
\includegraphics[width=\textwidth]{../Figure5/Figures/map.pdf}
\end{center}
\noindent{\bf Figure 6.} {\bf {\it A. thaliana} ancestry coeficients}. Ancestry coefficient estimate computed by the APLS algorithm with $K=6$ ancestral populations and $\sigma = 1.5$ for the range parameter A) Geographic map of ancestry coefficients. B) Barplot of ancestry coefficients.

\clearpage
\newpage

\begin{center}
\includegraphics[width=0.60\paperwidth]{../Figure5/Figures/manhattanplot.png}
%\includegraphics[width=\textwidth]{../Figure5/Figures/manhattanplot.pdf}
\end{center}
\noindent{\bf Figure 7.} {\bf Local adaptation in European lines of \bf {\it A.  thaliana} }. Manhattan plot of $-\log(p{\rm -value})$.  $p{\rm -value}$ were computed from population structure estimated by the APLS algorithm with $K=6$ ancestral populations and $\sigma = 1.5$ for the range parameter.

%\includegraphics[width=12cm]{Figures/Figure_2.pdf}
%\noindent{\bf Figure 2.} {\bf .}


\appendix

\section*{Acknowledgements}
This work has been partially supported by the LabEx PERSYVAL-Lab (ANR-11-LABX-0025-01) funded by the French program Investissement d\rq{}Avenir. Olivier Fran\c cois acknowledges support from Grenoble INP and from the Agence Nationale de la Recherche, project AFRICROP ANR-13-BSV7-0017.


% AOS,AOAS: If there are supplements please fill:
%\begin{supplement}[id=suppA]
%  \sname{Supplement A}
%  \stitle{Title}
%  \slink[doi]{10.1214/00-AOASXXXXSUPP}
%  \sdatatype{.pdf}" 
%  \sdescription{Some text}
%\end{supplement}

\clearpage
\newpage


\section*{References}

\begin{itemize}


\item[] Alexander DH, Lange K (2011) Enhancements to the ADMIXTURE algorithm for individual ancestry estimation. BMC bioinformatics, 12, 246.

\item[] Atwell S, Huang YS, Vilhj\"almsson BJ, et al (2010) Genome-wide association study of 107 phenotypes in {\it Arabidopsis thaliana} inbred lines. Nature, 465, 627-631.

\item[] Baran Y, Quintela I, Carracedo \`A, Pasaniuc B, Halperin E (2013) Enhanced localization of genetic samples through linkage-disequilibrium correction. The American Journal of Human Genetics, 92(6), 882-894.

\item[] Barton NH, Etheridge AM,  V?ber A (2010) A new model for evolution in a spatial continuum. Electron. J. Probab, 15(7), 162-216.

\item[] Bazin E, Dawson KJ, Beaumont MA (2010) Likelihood-free inference of population structure and local adaptation in a Bayesian hierarchical model. Genetics, 185, 587-602.

\item[] Belkin M, Niyogi P (2003) Laplacian eigenmaps for dimensionality reduction and data representation. Neural Computation, 15, 1373-1396.

\item[] Benjamini Y, Hochberg Y (1995) Controlling the false discovery rate: a practical and powerful approach to multiple testing. Journal of the Royal Statistical Society, 57, 289-300.

\item[] Bertsekas DP. {\it Nonlinear Programming}. Athena scientific, Belmont, USA, 1999.

\item[] Cai D, He X, Han J, Huang TS (2011) Graph regularized nonnegative matrix factorization for data representation. IEEE Transactions on Pattern Analysis and Machine Intelligence, 33, 1548-1560.

\item[] Cavalli-Sforza LL, Menozzi P, Piazza A (1994) {\it The History and Geography of Human Genes}. Princeton University Press, USA.

\item[] Caye K, Deist TM, Martins H, Michel O, Fran\c cois O (2015). TESS3: fast inference of spatial population structure and genome scans for selection. Molecular Ecology Resources 16 (2), 540-548.

\item[] Chen C, Durand E, Forbes F, Fran\c cois O (2007) Bayesian clustering algorithms ascertaining spatial population structure: a new computer program and a comparison study. Molecular Ecology Notes, 7, 747-756.

\item[]  Cichocki A, Zdunek R, Phan AH, Amari SI (2009) Nonnegative matrix and tensor factorizations: applications to exploratory multi-way data analysis and blind source separation. John Wiley and Sons, Chichester, UK.

\item[] Corander J, Sir\'en J,  Arjas E (2008) Bayesian spatial modeling of genetic population structure. Computational Statistics, 23(1), 111-129.

\item[] Chung FR (1997) {\it Spectral Graph Theory}. Vol. 92 of Regional Conference Series in Mathematics, American Mathematical Society, USA.

\item[] Cressie NAC (1993) {\it Statistics for Spatial Data, (Revised Edition)}. Wiley: New York, USA.

\item[] Devlin B, Roeder K (1999) Genomic control for association studies. Biometrics, 55, 997-1004.

\item[] Durand E, Jay F, Gaggiotti OE, Fran\c cois O (2009) Spatial inference of admixture proportions and secondary contact zones. Molecular Biology and Evolution, 26, 1963-1973.

\item[] Engelhardt BE, Stephens M (2010) Analysis of population structure: a unifying framework and novel methods based on sparse factor analysis. PLoS Genetics 6: 12.

\item[] Epperson BK (2003) {\it Geographical Genetics}. Princeton University Press, USA.


\item[] Fournier-Level A, et al. (2011) A map of local adaptation in {\it Arabidopsis thaliana}. Science. 334:86-89. 

\item[] Fran\c cois O, Ancelet S, Guillot G (2006) Bayesian clustering using hidden Markov random fields in spatial population genetics. Genetics, 174, 805-816.

\item[] Fran\c cois O, Blum MGB, Jakobsson M, Rosenberg NA (2008) Demographic history of European populations of {\it Arabidopsis thaliana}. PLoS Genetics, 4, e1000075.

\item[] Fran\c cois O, Durand E (2010) Spatially explicit Bayesian clustering models in population genetics. Molecular Ecology Resources, 10, 773-784.

\item[] Fran\c cois O,  Martins H, Caye K, Schoville SD (2016) Controlling false discoveries in genome scans for selection. Molecular Ecology 25 (2), 454-469

\item[] Fran\c cois O,  Waits LP (2016) Clustering and Assignment Methods in Landscape Genetics. In: Landscape Genetics (eds Balkenhol N, Cushman SA, Storfer AT, Waits LP), pp. 114-128. John Wiley and Sons, Ltd., Chichester, UK.

\item[] Frichot E, Schoville SD, Bouchard G, Fran\c cois O (2012) Correcting principal component maps for effects of spatial autocorrelation in population genetic data. Frontiers in Genetics, 3, 254.

\item[] Frichot E, Mathieu F, Trouillon T, Bouchard G, Fran\c cois O (2014) Fast and efficient estimation of individual ancestry coefficients. Genetics, 196, 973-983.

\item[] Frichot E, Fran\c cois O (2015) LEA: an R package for landscape and ecological association studies. Methods in Ecology and Evolution, 6, 925-929.

\item[] Hancock AM, et al. (2011) Adaptation to climate across the {\it Arabidopsis thaliana} genome. Science. 334:83-86.

\item[] Holsinger KE, Weir BS (2009) Genetics in geographically structured populations: defining, estimating and interpreting $F_{\rm ST}$. Nature Reviews Genetics, 10, 639-650.

\item[] Horton MW, Hancock AM, Huang YS, Toomajian C, Atwell S, Auton A, {\it et al.} (2012). Genome-wide patterns of genetic variation in worldwide {\it Arabidopsis thaliana} accessions from the RegMap panel. Nature genetics, 44(2), 212-216.

\item[] Hudson RR (2002) Generating samples under a Wright-Fisher neutral model of genetic variation. Bioinformatics, 18, 337-338.

\item[] Kelleher J, Etheridge AM, V\'eber A, Barton NH (2016). Spread of pedigree versus genetic ancestry in spatially distributed populations. Theoretical population biology, 108, 1-12.

\item[] Kim J, Park H (2011) Fast nonnegative matrix factorization: an active-set-like method and comparisons. SIAM Journal on Scientific Computing, 33, 3261-3281.

\item[] Kimura M, Weiss GH (1964) The stepping stone model of population structure and the decrease of genetic correlation with distance. Genetics, 49, 561.

\item[] Lao O, Liu F, Wollstein A,  Kayser M (2014) GAGA: a new algorithm for genomic inference of geographic ancestry reveals fine level population substructure in Europeans. PLoS Comput Biol, 10(2), e1003480.

\item[] Lee DD, Seung HS (1999) Learning the parts of objects by non-negative matrix factorization. Nature 401(6755): 788-791.


\item[] Mal\'ecot G (1948) {\it Les Math\'ematiques de l'H\'er\'edit\'e.} Masson, Paris.

\item[] Martins H, Caye K, Luu K,  Blum MGB, Francois O (2016) Identifying outlier loci in admixed and in continuous populations using ancestral population differentiation statistics. BioRxiv doi: http://dx.doi.org/10.1101/054585 

\item[] Popescu AA, Harper AL, Trick M, Bancroft I,  Huber KT (2014) A novel and fast approach for population structure inference using kernel-PCA and optimization. Genetics, 198(4), 1421-1431.

\item[] Pritchard JK, Stephens M, Donnelly P (2000) Inference of population structure using multilocus genotype data. Genetics, 155, 945-959.

\item[] Raj A, Stephens M, Pritchard JK (2014) fastSTRUCTURE: variational inference of population structure in large SNP data sets. Genetics, 197, 573-589.

\item[] Segelbacher G, Cushman SA, Epperson BK, et al (2010) Applications of landscape genetics in conservation biology: concepts and challenges. Conservation Genetics, 11, 375-385.

\item[] Schraiber JG, Akey JM (2015) Methods and models for unravelling human evolutionary history. Nature Reviews Genetics, 16, 727-740.

\item[] Tang H, Peng J, Wang P, Risch N (2005) Estimation of individual admixture: Analytical and study design considerations. Genet Epidemiol. 2005 28:289-301.

\item[] Weir (1996) {\it Genetic Data Analysis II}. Sinauer Associates Inc., Sunderland, MA.

\item[] Wollstein A, Lao O (2015) Detecting individual ancestry in the human genome. Investigative Genetics, 6, 1-12.

\item[] Wright S (1943) Isolation by distance. Genetics, 28, 114.

\item[] Yang WY, Platt A, Chiang CWK, Eskin E et al. (2014) Spatial localization of recent ancestors for admixed individuals. G3: Genes Genomes Genetics, 4(12), 2505-2518.

\item[] Carbon S, Ireland A, Mungall CJ, Shu S, Marshall B, Lewis S, AmiGO Hub, Web Presence Working Group. AmiGO: online access to ontology and annotation data. Bioinformatics. Jan 2009;25(2):288-9.

\end{itemize}



\end{document}
