\clearpage
\newpage


\section*{References}

\begin{itemize}


\item[] Alexander DH, Lange K (2011) Enhancements to the ADMIXTURE algorithm for individual ancestry estimation. BMC bioinformatics, 12, 246.

\item[] Atwell S, Huang YS, Vilhj\"almsson BJ, et al (2010) Genome-wide association study of 107 phenotypes in {\it Arabidopsis thaliana} inbred lines. Nature, 465, 627-631.

\item[] Baran Y, Quintela I, Carracedo \`A, Pasaniuc B, Halperin E (2013) Enhanced localization of genetic samples through linkage-disequilibrium correction. The American Journal of Human Genetics, 92(6), 882-894.

\item[] Barton NH, Etheridge AM,  V?ber A (2010) A new model for evolution in a spatial continuum. Electron. J. Probab, 15(7), 162-216.

\item[] Bazin E, Dawson KJ, Beaumont MA (2010) Likelihood-free inference of population structure and local adaptation in a Bayesian hierarchical model. Genetics, 185, 587-602.

\item[] Belkin M, Niyogi P (2003) Laplacian eigenmaps for dimensionality reduction and data representation. Neural Computation, 15, 1373-1396.

\item[] Benjamini Y, Hochberg Y (1995) Controlling the false discovery rate: a practical and powerful approach to multiple testing. Journal of the Royal Statistical Society, 57, 289-300.

\item[] Bertsekas DP. {\it Nonlinear Programming}. Athena scientific, Belmont, USA, 1999.

\item[] Cai D, He X, Han J, Huang TS (2011) Graph regularized nonnegative matrix factorization for data representation. IEEE Transactions on Pattern Analysis and Machine Intelligence, 33, 1548-1560.

\item[] Cavalli-Sforza LL, Menozzi P, Piazza A (1994) {\it The History and Geography of Human Genes}. Princeton University Press, USA.

\item[] Caye K, Deist TM, Martins H, Michel O, Fran\c cois O (2015). TESS3: fast inference of spatial population structure and genome scans for selection. Molecular Ecology Resources 16 (2), 540-548.

\item[] Chen C, Durand E, Forbes F, Fran\c cois O (2007) Bayesian clustering algorithms ascertaining spatial population structure: a new computer program and a comparison study. Molecular Ecology Notes, 7, 747-756.

\item[]  Cichocki A, Zdunek R, Phan AH, Amari SI (2009) Nonnegative matrix and tensor factorizations: applications to exploratory multi-way data analysis and blind source separation. John Wiley and Sons, Chichester, UK.

\item[] Corander J, Sir\'en J,  Arjas E (2008) Bayesian spatial modeling of genetic population structure. Computational Statistics, 23(1), 111-129.

\item[] Chung FR (1997) {\it Spectral Graph Theory}. Vol. 92 of Regional Conference Series in Mathematics, American Mathematical Society, USA.

\item[] Cressie NAC (1993) {\it Statistics for Spatial Data, (Revised Edition)}. Wiley: New York, USA.

\item[] Devlin B, Roeder K (1999) Genomic control for association studies. Biometrics, 55, 997-1004.

\item[] Durand E, Jay F, Gaggiotti OE, Fran\c cois O (2009) Spatial inference of admixture proportions and secondary contact zones. Molecular Biology and Evolution, 26, 1963-1973.

\item[] Engelhardt BE, Stephens M (2010) Analysis of population structure: a unifying framework and novel methods based on sparse factor analysis. PLoS Genetics 6: 12.

\item[] Epperson BK (2003) {\it Geographical Genetics}. Princeton University Press, USA.


\item[] Fournier-Level A, et al. (2011) A map of local adaptation in {\it Arabidopsis thaliana}. Science. 334:86-89. 

\item[] Fran\c cois O, Ancelet S, Guillot G (2006) Bayesian clustering using hidden Markov random fields in spatial population genetics. Genetics, 174, 805-816.

\item[] Fran\c cois O, Blum MGB, Jakobsson M, Rosenberg NA (2008) Demographic history of European populations of {\it Arabidopsis thaliana}. PLoS Genetics, 4, e1000075.

\item[] Fran\c cois O, Durand E (2010) Spatially explicit Bayesian clustering models in population genetics. Molecular Ecology Resources, 10, 773-784.

\item[] Fran\c cois O,  Martins H, Caye K, Schoville SD (2016) Controlling false discoveries in genome scans for selection. Molecular Ecology 25 (2), 454-469

\item[] Fran\c cois O,  Waits LP (2016) Clustering and Assignment Methods in Landscape Genetics. In: Landscape Genetics (eds Balkenhol N, Cushman SA, Storfer AT, Waits LP), pp. 114-128. John Wiley and Sons, Ltd., Chichester, UK.

\item[] Frichot E, Schoville SD, Bouchard G, Fran\c cois O (2012) Correcting principal component maps for effects of spatial autocorrelation in population genetic data. Frontiers in Genetics, 3, 254.

\item[] Frichot E, Mathieu F, Trouillon T, Bouchard G, Fran\c cois O (2014) Fast and efficient estimation of individual ancestry coefficients. Genetics, 196, 973-983.

\item[] Frichot E, Fran\c cois O (2015) LEA: an R package for landscape and ecological association studies. Methods in Ecology and Evolution, 6, 925-929.

\item[] Hancock AM, et al. (2011) Adaptation to climate across the {\it Arabidopsis thaliana} genome. Science. 334:83-86.

\item[] Holsinger KE, Weir BS (2009) Genetics in geographically structured populations: defining, estimating and interpreting $F_{\rm ST}$. Nature Reviews Genetics, 10, 639-650.

\item[] Horton MW, Hancock AM, Huang YS, Toomajian C, Atwell S, Auton A, {\it et al.} (2012). Genome-wide patterns of genetic variation in worldwide {\it Arabidopsis thaliana} accessions from the RegMap panel. Nature genetics, 44(2), 212-216.

\item[] Hudson RR (2002) Generating samples under a Wright-Fisher neutral model of genetic variation. Bioinformatics, 18, 337-338.

\item[] Kelleher J, Etheridge AM, V\'eber A, Barton NH (2016). Spread of pedigree versus genetic ancestry in spatially distributed populations. Theoretical population biology, 108, 1-12.

\item[] Kim J, Park H (2011) Fast nonnegative matrix factorization: an active-set-like method and comparisons. SIAM Journal on Scientific Computing, 33, 3261-3281.

\item[] Kimura M, Weiss GH (1964) The stepping stone model of population structure and the decrease of genetic correlation with distance. Genetics, 49, 561.

\item[] Lao O, Liu F, Wollstein A,  Kayser M (2014) GAGA: a new algorithm for genomic inference of geographic ancestry reveals fine level population substructure in Europeans. PLoS Comput Biol, 10(2), e1003480.

\item[] Lee DD, Seung HS (1999) Learning the parts of objects by non-negative matrix factorization. Nature 401(6755): 788-791.


\item[] Mal\'ecot G (1948) {\it Les Math\'ematiques de l'H\'er\'edit\'e.} Masson, Paris.

\item[] Martins H, Caye K, Luu K,  Blum MGB, Francois O (2016) Identifying outlier loci in admixed and in continuous populations using ancestral population differentiation statistics. BioRxiv doi: http://dx.doi.org/10.1101/054585 

\item[] Popescu AA, Harper AL, Trick M, Bancroft I,  Huber KT (2014) A novel and fast approach for population structure inference using kernel-PCA and optimization. Genetics, 198(4), 1421-1431.

\item[] Pritchard JK, Stephens M, Donnelly P (2000) Inference of population structure using multilocus genotype data. Genetics, 155, 945-959.

\item[] Raj A, Stephens M, Pritchard JK (2014) fastSTRUCTURE: variational inference of population structure in large SNP data sets. Genetics, 197, 573-589.

\item[] Segelbacher G, Cushman SA, Epperson BK, et al (2010) Applications of landscape genetics in conservation biology: concepts and challenges. Conservation Genetics, 11, 375-385.

\item[] Schraiber JG, Akey JM (2015) Methods and models for unravelling human evolutionary history. Nature Reviews Genetics, 16, 727-740.

\item[] Tang H, Peng J, Wang P, Risch N (2005) Estimation of individual admixture: Analytical and study design considerations. Genet Epidemiol. 2005 28:289-301.

\item[] Weir (1996) {\it Genetic Data Analysis II}. Sinauer Associates Inc., Sunderland, MA.

\item[] Wollstein A, Lao O (2015) Detecting individual ancestry in the human genome. Investigative Genetics, 6, 1-12.

\item[] Wright S (1943) Isolation by distance. Genetics, 28, 114.

\item[] Yang WY, Platt A, Chiang CWK, Eskin E et al. (2014) Spatial localization of recent ancestors for admixed individuals. G3: Genes Genomes Genetics, 4(12), 2505-2518.

\item[] Carbon S, Ireland A, Mungall CJ, Shu S, Marshall B, Lewis S, AmiGO Hub, Web Presence Working Group. AmiGO: online access to ontology and annotation data. Bioinformatics. Jan 2009;25(2):288-9.

\end{itemize}
