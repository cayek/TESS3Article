\section{Introduction}

 High-throughput sequencing technologies have enabled studies of genetic ancestry for model and non-model species at an unprecedented pace. In this  context, ancestry estimation algorithms are important for demographic analysis, medical genetics, conservation and landscape genetics (Pritchard et al. 2000, Tang et al. 2005, Schraiber and Akey 2015, Segelbacher et al. 2010, Fran\c cois and Waits 2016).  With increasingly large data sets, Bayesian approaches to the inference of population structure, exemplified by the computer program {\tt structure} (Pritchard et al. 2000), have been replaced by approximate algorithms that run several orders faster than the original version (Tang et al. 2005, Alexander and Lange 2011, Frichot et al. 2014, Raj et al. 2014).  Considering $K$ ancestral populations or genetic clusters, those algorithms estimate ancestry coefficients follow two main directions: model-based and model-free approaches. In model-based approaches, a likelihood function is defined for the matrix of ancestry coefficients, and estimation is performed by maximizing the logarithm of the likelihood function. For {\tt structure} and derived models, model assumptions include linkage equilibrium and Hardy-Weinberg equilibrium in ancestral populations. The first approximation to the original algorithm was based on an expectation-minimization algorithm (Tang et al. 2005), and more recent likelihood algorithms are implemented in the programs {\tt admixture}  and {\tt faststructure} (Alexander and Lange 2011, Raj et al. 2014). In model-free approaches, ancestry coefficients are estimated by using least-squares methods or factor analysis. Model-free methods make no assumptions about the biological processes that have generated the data. To estimate ancestry matrices, Engelhart and Stephens (2010) proposed to use sparse factor analysis, Frichot et al. (2014) used sparse non-negative matrix factorization algorithms, and Popescu et al. (2014) used kernel-principal component analysis. Least-squares methods accurately reproduce the results of likelihood approaches under the model assumptions of those methods (Frichot et al. 2014, Popescu et al. 2014).  In addition, model-free methods provide approaches that are valid when the assumptions of likelihood approaches are not met. Model-free methods are generally faster than model-based methods. 
   
  Among model-based approaches to ancestry estimation, an important class of methods have improved the Bayesian model of {\tt structure} by incorporating geographic data through spatially informative prior distributions (Chen et al. 2007, Corander et al. 2008). Under isolation-by-distance patterns (Wright 1943, Mal\'ecot 1948), spatial algorithms provide more robust estimates of population structure than non-spatial algorithms which can lead to biased estimates of the number of clusters (Durand et al. 2009).  Bayesian methods are based on Markov chain Monte Carlo algorithms which are computer-intensive (Fran\c cois and Durand 2010). Recent efforts to improve the inference of ancestral relationships in a geographical context have mainly focused on the localization of recent ancestors (Baran et al. 2013, Lao et al. 2014, Yang et al. 2014). In these applications, spatial information is used in a predictive framework that assigns ancestors to putative geographic origins.  While geographic estimation of individual ancestry proportions has been proposed in the program {\tt tess3}, here is a growing need to develop fast individual ancestry estimation algorithms that reduce computational cost in a geographically explicit framework (Caye et al. 2016). 

In this study, we present two new algorithms for the estimation of ancestry matrices based on geographic and genetic data based on Alternating Projected Least Squares (APLS) and Alternating Quadratic Programming  (AQP). While AQP algorithms have a well-established theoretical background (Bertsekas 1999), this is not the case of APLS algorithms. Using coalescent simulations, we provide evidence that the estimates computed by APLS algorithms are good approximations to the solutions of AQP algorithms. In addition, we show that the performances of APLS algorithms scale with the dimensions of modern data sets, and outperform those of previously developed algorithms. We discuss the application of our algorithms to data from European ecotypes of {\it Arabidopsis thaliana}, for which individual genomic an geographic data are available (Horton et al. 2012). 



