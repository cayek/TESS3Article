\section{Introduction}

 High-throughput sequencing technologies have enabled studies of genetic ancestry for model and non-model species at an unprecedented pace. Developing ancestry estimation methods is important for demographic analysis, medical genetics, conservation genetics or landscape genetics (Pritchard et al. 2000, Tang et al. 2005, Schraiber and Akey 2015, Segelbacher et al. 2010, Fran\c cois and Waits 2016).  With large data sets, Bayesian approaches to the inference of population structure, exemplified by the computer program {\tt structure} (Pritchard et al. 2000), have been replaced by approximate algorithms that run several orders faster than the original version (Tang et al. 2005, Alexander and Lange 2011, Frichot et al. 2014, Raj et al. 2014).  Those modern approaches to the estimation of ancestry coefficients can be  divided into two main branches: model-based and model-free approaches, considering $K$ ancestral populations ($K$ is unknown). In model-based approaches, a likelihood function is defined for the matrix of ancestry coefficients, and estimation is performed by maximizing the logarithm of the likelihood function. For {\tt structure}, the model assumptions include linkage equilibrium and Hardy-Weinberg equilibrium in ancestral populations. The first fast approximation to the original algorithm was based on an expectation-minimization algorithm (Tang et al. 2005). Other fast algorithms are implemented in the programs {\tt admixture}  and {\tt faststructure} (Alexander and Lange 2011, Raj et al. 2014). In model-free approaches, ancestry coefficients are estimated by using least-squares methods or factor analysis. Model-free methods make no assumptions about the biological processes that have generated the data. To estimate ancestry matrices, Engelhart and Stephens (2010) proposed to use sparse factor analysis. Frichot et al. (2014) used sparse non-negative matrix factorization algorithms, and Popescu et al. (2014) used kernel-principal component analysis. Least-squares methods accurately reproduce the results of likelihood approaches under the model assumptions of those methods (Frichot et al. 2014, Popescu et al. 2014).  In addition, model-free methods provide approaches that are valid when the assumptions of likelihood approaches are not met, and they are generally faster than the model-based approaches. 
   
  Among model-based approaches to ancestry estimation, an important class of methods have improved the Bayesian model of {\tt structure} by incorporating geographic data in the prior distributions  (Chen et al. 2007, Corander et al. 2008). Under isolation-by-distance patterns (Wright 1943, Mal\'ecot 1948), spatial algorithms can provide more robust estimates of population structure than non-spatial algorithms which lead to biased estimates of the number of clusters (Durand et al. 2009).  Most spatial methods are based on Markov chain Monte Carlo algorithms which are computer-intensive (Fran\c cois and Durand 2010). While a recent geographic approach to the estimation individual ancestry proportions has been implemented in the program {\tt tess3}  (Caye et al. 2016),  efforts to improve the inference of ancestral relationships in a geographical context have mainly focused on the localization of recent ancestors (Baran et al. 2013, Lao et al. 2014, Yang et al. 2014). In these applications, spatial information is used in a predictive framework to assign ancestors to putative geographic origins. There is a growing need to develop fast individual ancestry estimation algorithms that reduce the computational burden in a geographically explicit framework. 

In this study, we present two new algorithms for the estimation of ancestry matrices based on geographic and genetic data: Alternating Quadratic Programming  (AQP) and Alternating Projected Least Squares (APLS). We show that the solutions found by APLS algorithms provide good approximations for the solutions of AQP algorithms, which have a well-established theoretical background (Bertsekas 1999). Using coalescent simulations, we provide evidence that the performances of APLS algorithms outperform those implemented in previous programs, and that their runtimes scale with the dimensions of modern data sets. We discuss the application of the new algorithms to European {\it Arabidopsis thaliana} ecotypes, for which individual geographic data are available (Horton et al. 2012). 



