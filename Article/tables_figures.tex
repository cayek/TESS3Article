\clearpage
\newpage
  
\begin{center}
\begin{table}
\caption{{\bf }}
\begin{tabular}{lccc}
\hline
{\bf } &   Sample size & Number of SNPs &  Reference \\
\hline
 &  &   & \\

  &  &   & \\
 \hline
\end{tabular}
\end{table}
\end{center}        




\clearpage 
\newpage

%\section*{Figure legends}

\begin{center}
\includegraphics{../Figure1/Figures/figure1.pdf}
\end{center}  
\noindent{\bf Figure 1.} {\bf Root Mean Squared Errors (RMSEs) for the $Q$ and $G$ matrix estimates.} Simulations of spatially admixed populations. A) Statistical errors for APLS, AQP and {\tt tess3} estimates as a function of the sample  size, $n$ ($L = $). B) Statistical errors for APLS, AQP and {\tt tess3} estimates as a function of the number of loci, $L$ ($n = 200$).



MANQUE les nombres fixes n,L dans A et B.

\clearpage 
\newpage


\begin{center}
\includegraphics{../Figure2/Figures/figure2.pdf}
\end{center}  
\noindent{\bf Figure 2.} {\bf } 

\clearpage 
\newpage

\begin{center}
\includegraphics{../Figure3/Figures/figure3.pdf}
\end{center}
\noindent{\bf Figure 3.} {\bf Area under the precision-recall curve (AUC)}. Neutrality tests applied to simulations of spatially admixed populations. AUCs for tests based on $F_{\rm ST}$ with the true ancestral populations,  spatial ancestry estimates computed with APLS algorithms, non-spatial ({\tt structure}-like) ancestry estimates computed with the {\tt snmf} algorithm. The relative intensity of selection in ancestral populations, defined as the ratio $m/m_s$, was varied in the range $1-160$.


\clearpage 
\newpage

\begin{center}
\includegraphics{../Figure4/Figures/figure4.pdf}
\end{center}
\noindent{\bf Figure 4.} {\bf Numbers of iteration and runtimes for the AQP, APLS and {\tt tess3} algorithm implementations}. A-B)   Total number of iterations before an algorithm reached a steady solution. C-D) Runtime for a single iteration (seconds). The number of SNPs was kept fixed to $L = 50$k in A and C. The number of individuals was kept fixed to $n = 150$ in B and D. 


\clearpage 
\newpage

\begin{center}
\includegraphics{../Figure5/Figures/figure5.pdf}
\end{center}
\noindent{\bf Figure 5.} 

\clearpage 
\newpage

\begin{center}
\includegraphics{../Figure5/Figures/map.pdf}
\end{center}
\noindent{\bf Figure 6.} 

\clearpage 
\newpage

\begin{center}
\includegraphics{../Figure5/Figures/map.R.pdf}
\end{center}
\noindent{\bf Figure 7.} 

%\includegraphics[width=12cm]{Figures/Figure_2.pdf}
%\noindent{\bf Figure 2.} {\bf .} 


